\documentclass{abntex2}
\usepackage[utf8]{inputenc}
\usepackage[T1]{fontenc}
\usepackage[brazil]{babel}
\usepackage{lmodern}
\usepackage{hyperref}
\usepackage{graphicx}
\usepackage{amsmath}
\usepackage{amssymb}
\usepackage{float}
\usepackage{listings}
\usepackage{color}
\usepackage{caption}
\usepackage{subcaption}
\usepackage{tabularx}
\usepackage{longtable}
\usepackage{booktabs}
\usepackage{array}
\usepackage{fancyhdr}
\usepackage{geometry}
\geometry{a4paper, margin=2.5cm}
\hypersetup{
    colorlinks=true,
    linkcolor=blue,
    filecolor=magenta,
    urlcolor=cyan,
    pdftitle={Guia Prático para a Capacitação em GitHub},
    pdfpagemode=FullScreen,
}
\title{Guia Prático para a Capacitação em GitHub}
\author{Ronivaldo Domingues de Andrade}
\date{\today}
\begin{document}
\maketitle
\newpage
\tableofcontents
\newpage
\section{Introdução}
Este guia prático tem como objetivo fornecer uma introdução abrangente ao uso do GitHub, uma plataforma popular para controle de versão e colaboração em projetos de software. O GitHub é amplamente utilizado por desenvolvedores, equipes de desenvolvimento e comunidades open source para gerenciar código-fonte, acompanhar mudanças e colaborar em projetos de forma eficiente.
\section{Criando uma Conta no GitHub}
Para começar a usar o GitHub, você precisa criar uma conta. Siga os passos abaixo:
\begin{enumerate}
    \item Acesse o site do GitHub: \url{https://github.com/}
\end{enumerate}
\section{Principais Funcionalidades do GitHub}
O GitHub oferece uma variedade de funcionalidades que facilitam o desenvolvimento colaborativo. Algumas das principais funcionalidades incluem:
\begin{itemize}
    \item Repositórios: Armazenamento de código-fonte e arquivos relacionados.
    \item Branches: Permitem o desenvolvimento paralelo e a experimentação.
    \item Pull Requests: Facilitam a revisão e a integração de mudanças.
    \item Issues: Sistema de rastreamento de bugs e tarefas.
    \item Actions: Automação de fluxos de trabalho.
\end{itemize}
\section{Colaboração em Projetos}
O GitHub é uma plataforma colaborativa que permite que várias pessoas trabalhem juntas em um projeto. Aqui estão algumas dicas para colaborar efetivamente:
\begin{itemize}
    \item Use branches para desenvolver novas funcionalidades ou corrigir bugs sem afetar o código principal.
    \item Crie pull requests para revisar e discutir mudanças antes de integrá-las ao código principal.
    \item Utilize issues para rastrear bugs, solicitar novas funcionalidades e organizar tarefas.
    \item Comente no código e nas pull requests para fornecer feedback construtivo.
\end{itemize}
\section{Recursos Adicionais}
Para aprofundar seus conhecimentos sobre o GitHub, considere explorar os seguintes recursos:
\begin{itemize}
    \item Documentação oficial do GitHub: \url{https://docs.github.com/}
    \item Tutoriais e cursos online sobre Git e GitHub.
    \item Comunidades e fóruns de desenvolvedores para compartilhar experiências e aprender com outros.
\end{itemize}
\section{Conclusão}
O GitHub é uma ferramenta poderosa para o desenvolvimento colaborativo de software. Com este guia prático, você está pronto para começar a explorar e utilizar o GitHub em seus projetos. Lembre-se de praticar regularmente e aproveitar os recursos disponíveis para aprimorar suas habilidades.
\newpage
\section{Lista de Presenças}
\noindent
\begin{tabularx}{\textwidth}{|>{\centering\arraybackslash}X|>{\centering\arraybackslash}X|}
\hline
\textbf{Nome} & \textbf{Presente em} \\
\hline
\end{tabularx}
\end{document}