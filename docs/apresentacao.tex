% ====== Definindo a classe do documento ======%
\documentclass[10pt, brazil]{beamer}
% ====== Importando os pacotes a serem usados ======%
%\usepackage[utf8]{inputenc}
\usepackage[brazil]{babel} % Traduz inserções automáticas para o português como por exemplo a saída do comando "\today", que por padrão é em inglẽs;
\usepackage[T1]{fontenc}
\usepackage{xcolor}
\usepackage{graphicx}
\usepackage{bookmark}
\usepackage{amsmath, amssymb, amsfonts}
\usepackage{tikz}
\usepackage{animate}
% == Configurando a fonte Lexend Localmente e por isso esse documento deve ser compilado com o xelatex e não com o pdflatex ==%
\usepackage{fontspec} % Pacote para definir fontes personalizadas;
\newcommand{\fontpath}{./assets/fonts/} % Caminho dos arquivos estaticos da fonte;
% ==== Define a fonte principal (Regular e Bold) ====%
\setmainfont[
    Path=\fontpath,
    UprightFont = *-Regular,
    BoldFont = *-ExtraBold,
]{Lexend}
% ==== Define variantes para serem usadas manualmente ====%
\newfontfamily\lexendsemibold[
    Path=\fontpath,
    UprightFont = *-SemiBold,
]{Lexend}
\newfontfamily\lexendlight[
    Path=\fontpath,
    UprightFont = *-Light,
]{Lexend}
% =======================================Tema personalizado de acordo com o manual da marca for_code==============================%
\definecolor{darkpurple}{HTML}{1E1647}
\definecolor{lightpurple}{HTML}{614AD3}
\definecolor{yellow}{HTML}{FFFF00}
\definecolor{white}{HTML}{FFFFFF}
% =======================================================Definindo Tema do Slide==================================================%
\setbeamercolor{background canvas}{bg=darkpurple}
\setbeamercolor{normal text}{fg=white}
\setbeamercolor{structure}{fg=lightpurple}
\setbeamercolor{itemize item}{fg=lightpurple}
% ================================Configurando o Rodapé================================%
\setbeamertemplate{footline}{
  \leavevmode%
  \vspace{1em}
  \makebox[\paperwidth]{%
    \hspace{1em}
    \includegraphics[width=0.15\linewidth]{./assets/logos/for_code_para_fudo_escuro.png}%
    \hfill
    \textcolor{lightpurple}{\scriptsize \textbf{[ \insertframenumber \;]}}%
    \hspace{1em}
  }%
  \vspace{0.5em}
}
\setbeamertemplate{navigation symbols}{}
% ===============================Configurando o Título=================================%
\title{{\Huge \textcolor{lightpurple}{\textbf{[}} \textbf{Capacitação GitHub} \textcolor{lightpurple}{\textbf{]}}}}
\author{{\normalsize Ronivaldo D. Andrade - Vice Diretor de Projetos }}
\date{{\normalsize \today}}
% === Ativar anotações do apresentador ===%
\usepackage{pgfpages}
% === Escolha como as notas serão exibidas ===%
% === Durante a apresentação: apenas você vê as notas no modo dual-screen ===%
% === Para gerar PDF com as notas no lado esquerdo de cada slide, descomente: ===%
%
%\setbeameroption{show notes on second screen=left}
%
% === Para gerar PDF com as notas no lado direito de cada slide, descomente: ===%
%
%\setbeameroption{show notes on second screen=right}
%
% === Para gerar PDF com as notas acima de cada slides: ===%
%
%\setbeameroption{show notes on second screen=top}
%
% === Para gerar PDF com as notas abaixo de cada slides: ===%
%
%\setbeameroption{show notes on second screen=bottom}
%
% === Para gerar PDF somente com as notas: ===%
%
%\setbeameroption{show only notes}
%
% ====== Inserir imagem fixa em todos os slides ====== %
\addtobeamertemplate{background}{
  \ifnum\insertframenumber>2
    \ifnum\insertframenumber<\inserttotalframenumber
      \begin{tikzpicture}[remember picture, overlay]
        \node[anchor=center] at ([xshift=4cm, yshift=-2.3cm]current page.center) {
          \includegraphics[width=0.18\paperwidth]{./assets/logos/simbolo_para_fundo_escuro.png}
          %\includegraphics[width=0.18\paperwidth]{example-image-a} % Retangulo Teste
        };
      \end{tikzpicture}
    \fi
  \fi
}{}
% ====== Cores para os alert blocks ======%
\setbeamercolor{alerted text}{fg=yellow}
\setbeamercolor{alertblock title}{bg=yellow, fg=darkpurple}
\setbeamercolor{alertblock body}{bg=yellow, fg=darkpurple}
%
\setbeamercolor{exampleblock title}{bg=lightpurple, fg=white}
\setbeamercolor{exampleblock body}{bg=lightpurple, fg=white}
%
\setbeamercolor{block title example}{bg=lightpurple, fg=white}
\setbeamercolor{block body example}{bg=lightpurple, fg=white}
% ====== Configurar marcadores de lista ======%
\setbeamertemplate{itemize item}{\textcolor{lightpurple}{\rule[0.3pt]{6pt}{6pt}}}
\setbeamertemplate{itemize subitem}{\textcolor{lightpurple}{\rule[0.3pt]{5pt}{5pt}}}
\setbeamertemplate{itemize subsubitem}{\textcolor{lightpurple}{\rule[0.3pt]{4pt}{4pt}}}
% ======================================================================Início do Documento==============================================%
\begin{document}
% =========== Slide 0 ===========%
\begin{frame}[plain]{\textcolor{lightpurple}{\textbf{[}} \textbf{\textcolor{white}{Uma breve citação para começar}} \textcolor{lightpurple}{\textbf{]}}}

\vspace{3em}

\begin{center}
  \lexendlight
  \textcolor{white}{
    \Large
    “Isso aqui vai ser uma \textcolor{yellow}{palestra} de \textcolor{lightpurple}{GitHub intermediário}.”
  }

  \vspace{1em}

  \textcolor{lightpurple}{
    \small
    — Um sábio contemporâneo (2025)
  }
\end{center}

\vfill

\begin{flushright}
  \textcolor{lightpurple}{\scriptsize *Paráfrase livre de uma lenda que se foi...*}
\end{flushright}

\end{frame}
% =========== Slide 1 ===========%

\begin{frame}[plain]
  \vspace{2em}
  \includegraphics[width=0.18\linewidth]{./assets/logos/for_code_para_fudo_escuro.png}

  \vfill
  \begin{center}
    {\usebeamerfont{title}\inserttitle}

    \vspace{1em}
    {\textcolor{lightpurple}{\insertauthor – \insertdate}}
  \end{center}

  \vfill
  \begin{flushright}
    \textcolor{lightpurple}{\rule{0.3\linewidth}{3pt}}
  \end{flushright}
  
  \note{
		Slide 01:\\
		\textbf{Boa noite a todos!}
	}
\end{frame}
% =========== Slide 2 ===========%
\begin{frame}{\textcolor{lightpurple}{\textbf{[}} \textbf{\textcolor{white}{O que é GitHub?}} \textcolor{lightpurple}{\textbf{]}}}
  
  \begin{columns}
    \begin{column}{0.6\textwidth}
      \begin{itemize}
        \item Plataforma de hospedagem de código-fonte
        \item Utiliza sistema de controle de versão Git
        \item Permite colaboração em projetos
        \item Oferece recursos como:
        \begin{itemize}
          \item GitHub Pages
          \item GitHub Actions
          \item Pull Requests
          \item Code Review
        \end{itemize}
      \end{itemize}
    \end{column}
    % \begin{column}{0.4\textwidth}
    %   \centering
    %   \includegraphics[width=0.8\linewidth]{./logos/github-logo.png}
    % \end{column}
  \end{columns}
  \begin{alertblock}{Mão na massa!}
    No Capítulo 2 do Guia, a partir do Item 2.2,\\
    temos um guia resumido de como criar sua\\
    primeira conta no GitHub.
  \end{alertblock}

  \note{
    Slide 02:\\
    \textbf{O GitHub é muito mais que um simples repositório de código.}\\
    É uma plataforma completa que permite desenvolvedores colaborarem\\
    de forma eficiente, com ferramentas poderosas para versionamento,\\
    automação e deploy de projetos.
  }
\end{frame}

% =========== Slide 3 ===========%
\begin{frame}{\textcolor{lightpurple}{\textbf{[}} \textbf{\textcolor{white}{GitHub Student Developer Pack}} \textcolor{lightpurple}{\textbf{]}}}
  
  \begin{block}{Benefícios para Estudantes}
    \begin{itemize}
      \item \textbf{Conta GitHub Pro gratuita}
      \item \textbf{3.000 minutos} do GitHub Actions (vs 2.000)
      \item \textbf{2GB} de armazenamento (vs 500MB)
      \item \textbf{GitHub Copilot} incluso
      \item \textbf{180 horas} de Codespaces (vs 120)
      \item \textcolor{yellow}{\textbf{Confira o Guia, Capítulo 2, Item 2.2.1 para maiores informações.}}
    \end{itemize}
  \end{block}

  \vspace{1em}
  
  \begin{alertblock}{Como obter}
    \begin{enumerate}
      \item Adicionar e-mail acadêmico
      \item Acessar education.github.com/pack
      \item Enviar comprovante de matrícula
      \item Aguardar a análise do documento e a aprovação
    \end{enumerate}
  \end{alertblock}

  \note{
    Slide 03:\\
    \textbf{O GitHub Student Developer Pack é um programa incrível}\\
    que oferece benefícios de conta Pro gratuitamente para estudantes.\\
    Com ele você tem mais recursos para CI/CD, armazenamento\\
    e até o GitHub Copilot para ajudar na programação.
  }
\end{frame}

% =========== Slide 4 ===========%
\begin{frame}{\textcolor{lightpurple}{\textbf{[}} \textbf{\textcolor{white}{Instalação e Configuração}} \textcolor{lightpurple}{\textbf{]}}}
  
  \begin{columns}
    \begin{column}{0.5\textwidth}
      \textbf{Ferramentas Necessárias:}
      \begin{itemize}
        \item \textcolor{yellow}{Winget} - Gerenciador de pacotes
        \item \textcolor{yellow}{Git} - Controle de versão
        \item \textcolor{yellow}{GitHub CLI} - Interface de linha de comando
      \end{itemize}
    \end{column}
    \begin{column}{0.5\textwidth}
      \textbf{Comandos de Instalação:}
      \begin{scriptsize}
        \begin{itemize}
          \item \texttt{winget install Git.Git}
          \item \texttt{winget install GitHub.cli}
          \item \texttt{gh auth login}
        \end{itemize}
      \end{scriptsize}
    \end{column}
  \end{columns}

  \vspace{1em}

  \begin{block}{Configuração Básica do Git}
    \begin{scriptsize}
      \texttt{git config --global user.name "Seu Nome"}\\
      \texttt{git config --global user.email "seu@email.com"}
    \end{scriptsize}
    \begin{center}
      \textcolor{yellow}{\textbf{Confira o Guia, Capítulo 3 e 4\\
      para maiores informações.}}
    \end{center}
  \end{block}

  \note{
    Slide 04:\\
    \textbf{Vamos instalar as ferramentas essenciais usando o Winget.}\\
    Configurem o Git com nome e e-mail - isso aparecerá nos commits.\\
    Autentiquem no GitHub CLI, é mais prático que ficar digitando token.
  }
\end{frame}

% =========== Slide 5 ===========%
\begin{frame}{\textcolor{lightpurple}{\textbf{[}} \textbf{\textcolor{white}{Comandos Git Essenciais}} \textcolor{lightpurple}{\textbf{]}}}
  
  \begin{columns}
    \begin{column}{0.5\textwidth}
      \textbf{Comandos Básicos:}
      \begin{itemize}
        \item \texttt{git init} - Inicializa repositório
        \item \texttt{git clone} - Clona repositório remoto
        \item \texttt{git add} - Adiciona arquivos ao stage
        \item \texttt{git commit} - Salva alterações
        \item \texttt{git push} - Envia para o remoto
        \item \texttt{git pull} - Atualiza do remoto
      \end{itemize}
    \end{column}
    \begin{column}{0.5\textwidth}
      \textbf{Fluxo Básico:}
      \begin{enumerate}
        \item \texttt{git add .}
        \item \texttt{git commit -m "msg"}
        \item \texttt{git push}
      \end{enumerate}
      
      \vspace{1em}
      \textbf{Branching:}
      \begin{itemize}
        \item \texttt{git branch}
        \item \texttt{git checkout -b}
        \item \texttt{git merge}
      \end{itemize}
    \end{column}
  \end{columns}

  \note{
    Slide 05:\\
    \textbf{Estes são os comandos Git que vocês vão usar diariamente.}\\
    O fluxo básico é: adicionar mudanças, commitar com mensagem clara,\\
    e enviar para o repositório remoto. Trabalhem com branches\\
    para organizar features diferentes.
  }
\end{frame}

% =========== Slide 6 ===========%
\begin{frame}{\textcolor{lightpurple}{\textbf{[}} \textbf{\textcolor{white}{Git LFS - Arquivos Grandes}} \textcolor{lightpurple}{\textbf{]}}}
  
  \begin{block}{O que é Git LFS?}
    Extensão do Git para versionar arquivos grandes (imagens, vídeos, modelos)
  \end{block}

  \vspace{1em}

  \begin{itemize}
    \item \textbf{Problema:} Git tradicional não lida bem com arquivos >100MB
    \item \textbf{Solução:} Armazena apenas ponteiros no repositório
    \item \textbf{Conteúdo real} fica em servidor separado
  \end{itemize}

  \vspace{1em}

  \begin{block}{Como usar}
    \begin{scriptsize}
      \texttt{git lfs install}\\
      \texttt{git lfs track "*.psd"}\\
      \texttt{git lfs track "*.zip"}\\
      \texttt{git add .gitattributes}\\
      \texttt{git add arquivo-grande.zip}
    \end{scriptsize}
  \end{block}

  \note{
    Slide 06:\\
    \textbf{Git LFS resolve um problema comum: arquivos grandes.}\\
    Em vez de armazenar o arquivo inteiro no histórico,\\
    ele guarda apenas um ponteiro, mantendo o repositório leve.\\
    Essencial para projetos com assets pesados.
  }
\end{frame}

% =========== Slide 7 ===========%
\begin{frame}{\textcolor{lightpurple}{\textbf{[}} \textbf{\textcolor{white}{Commits Profissionais}} \textcolor{lightpurple}{\textbf{]}}}
  
  \begin{block}{Boas Práticas}
    \begin{itemize}
      \item \textbf{Commits atômicos} - Uma mudança lógica por commit
      \item \textbf{Mensagens claras} - Assunto + descrição (se necessário)
      \item \textbf{Modo imperativo} - "Adiciona", "Corrige", "Remove"
      \item \textbf{Referência a issues} - "Closes \#42", "Fixes \#123"
    \end{itemize}
  \end{block}

  \vspace{1em}

  \begin{block}{Exemplo de Commit}
    \begin{scriptsize}
      git commit -m " $<$tipo>(escopo): <descrição>"\\
      \textbf{git commit -m "feat(login): adiciona autenticação via OAuth2"}\\
      \ \\
      Implementa sistema de login usando OAuth2 do Google.\\
      Inclui validação de tokens e tratamento de erros.\\
      Closes \#15
    \end{scriptsize}
  \end{block}

  \note{
    Slide 07:\\
    \textbf{Commits bem feitos facilitam a vida de toda a equipe.}\\
    Façam commits pequenos e focados, com mensagens claras\\
    que expliquem O QUE foi feito e POR QUE foi feito.\\
    Usem padrões como Conventional Commits.
  }
\end{frame}

% =========== Slide 8 ===========%
\begin{frame}{\textcolor{lightpurple}{\textbf{[}} \textbf{\textcolor{white}{Merge vs Rebase}} \textcolor{lightpurple}{\textbf{]}}}
  
  \begin{columns}
    \begin{column}{0.5\textwidth}
      \textbf{Merge}
      \begin{itemize}
        \item Preserva histórico completo
        \item Cria commit de merge
        \item Mais seguro para histórico compartilhado
        \item \texttt{git merge feature-branch}
      \end{itemize}
    \end{column}
    \begin{column}{0.5\textwidth}
      \textbf{Rebase}
      \begin{itemize}
        \item Histórico linear e limpo
        \item Reescreve histórico
        \item Ideal para branches locais
        \item \texttt{git rebase main}
      \end{itemize}
    \end{column}
  \end{columns}

  \vspace{1em}

  \begin{alertblock}{Importante!}
    \begin{itemize}
      \item \textbf{Nunca} faça rebase em histórico compartilhado
      \item Use \texttt{--force-with-lease} em vez de \texttt{--force}
      \item Combine com a equipe qual estratégia usar
    \end{itemize}
  \end{alertblock}

  \note{
    Slide 08:\\
    \textbf{Merge e Rebase são duas formas de integrar branches.}\\
    Merge é mais seguro e preserva todo o histórico.\\
    Rebase deixa o histórico mais limpo mas reescreve commits.\\
    Cuidado com rebase em branches compartilhadas!
  }
\end{frame}

% =========== Slide 9 ===========%
\begin{frame}{\textcolor{lightpurple}{\textbf{[}} \textbf{\textcolor{white}{Pull Requests}} \textcolor{lightpurple}{\textbf{]}}}
  
  \begin{block}{O que são Pull Requests?}
    Solicitação formal para integrar mudanças de uma branch em outra, com revisão de código
  \end{block}

  \vspace{1em}

  \textbf{Fluxo Básico:}
  \begin{enumerate}
    \item Criar branch para feature/bugfix
    \item Desenvolver e commitar localmente
    \item Fazer push da branch
    \item Criar PR no GitHub
    \item Revisão e aprovação
    \item Merge no branch principal
  \end{enumerate}

  \vspace{1em}

  \begin{block}{Via GitHub CLI}
    \texttt{gh pr create --title $``$Minha feature$"$ --body $``$Descrição$"$}
  \end{block}

  \note{
    Slide 09:\\
    \textbf{Pull Requests são o coração da colaboração no GitHub.}\\
    Permitem revisão de código, discussão sobre implementação\\
    e integração controlada de mudanças.\\
    Podem ser criados tanto pela web quanto pela CLI.
  }
\end{frame}

% =========== Slide 10 ===========%
\begin{frame}{\textcolor{lightpurple}{\textbf{[}} \textbf{\textcolor{white}{GitHub Pages \& Actions}} \textcolor{lightpurple}{\textbf{]}}}
  
  \begin{columns}
    \begin{column}{0.5\textwidth}
      \textbf{GitHub Pages}
      \begin{itemize}
        \item Hospedagem gratuita de sites estáticos
        \item Suporte a domínios customizados
        \item Integração com Jekyll, Hugo, etc.
        \item Atualização automática
      \end{itemize}
    \end{column}
    \begin{column}{0.5\textwidth}
      \textbf{GitHub Actions}
      \begin{itemize}
        \item Automação de CI/CD
        \item Testes automatizados
        \item Deploy automático
        \item Workflows customizáveis
      \end{itemize}
    \end{column}
  \end{columns}

  \vspace{1em}

  \begin{block}{Exemplo de Integração}
    \begin{scriptsize}
      Push na main $\rightarrow$ Tests $\rightarrow$ Build $\rightarrow$ Deploy no GitHub Pages
    \end{scriptsize}
  \end{block}

  \note{
    Slide 10:\\
    \textbf{GitHub Pages e Actions são recursos poderosos.}\\
    Pages permite hospedar sites diretamente do repositório.\\
    Actions automatiza testes, builds e deploys.\\
    Juntos, criam um pipeline completo de desenvolvimento.
  }
\end{frame}

% =========== Slide 11 ===========%
\begin{frame}{\textcolor{lightpurple}{\textbf{[}} \textbf{\textcolor{white}{Assinatura GPG de Commits}} \textcolor{lightpurple}{\textbf{]}}}
  
  \begin{block}{Por que assinar commits?}
    \begin{itemize}
      \item \textbf{Autenticidade} - Confirma que você fez o commit
      \item \textbf{Integridade} - Garante que não foi alterado
      \item \textbf{Verificação} - Mostra "Verified" no GitHub
    \end{itemize}
  \end{block}

  \vspace{1em}

  \textbf{Configuração:}
  \begin{enumerate}
    \item Gerar chave GPG: \texttt{gpg --full-generate-key}
    \item Configurar Git: \texttt{git config --global user.signingkey ID}
    \item Adicionar chave pública no GitHub
    \item Commitar com \texttt{git commit -S -m "msg"}
  \end{enumerate}

  \note{
    Slide 11:\\
    \textbf{Assinar commits com GPG adiciona uma camada de segurança.}\\
    Garante que os commits foram realmente feitos por você\\
    e não foram adulterados. No GitHub, aparecem como "Verified".\\
    É uma prática profissional importante.
  }
\end{frame}

% =========== Slide 12 ===========%
\begin{frame}{\textcolor{lightpurple}{\textbf{[}} \textbf{\textcolor{white}{Exercícios Práticos}} \textcolor{lightpurple}{\textbf{]}}}
  
  \begin{block}{O que vamos praticar?}
    \begin{itemize}
      \item Criar repositório via GitHub CLI
      \item Clonar e fazer mudanças
      \item Trabalhar com branches
      \item Fazer Pull Requests
      \item Configurar GitHub Pages
      \item Criar workflow básico no Actions
    \end{itemize}
  \end{block}

  \vspace{1em}

  \begin{alertblock}{Mão na massa!}
    Agora acessem o capitulo 9 do Guia que preparei,\\ pois lá tem o passo a passo para alguns exercícios.
  \end{alertblock}

  \note{
    Slide 12:\\
    \textbf{Agora é hora de colocar a mão na massa!}\\
    Vamos praticar desde a criação do repositório\\
    até deploy automático com GitHub Pages.\\
    Preparem suas máquinas e vamos codar!
  }
\end{frame}
% =========== Slide Final ===========%
\begin{frame}[plain]
	\vspace{2em}
	\includegraphics[width=0.18\linewidth]{./assets/logos/for_code_para_fudo_escuro.png}
	
	\vfill
	\textbf{{\fontsize{40}{48}\selectfont Obrigado!}} \\
	\vspace{2em}
	\textbf{\textcolor{lightpurple}{{\large 
	Capacitação em GitHub \\
	\vspace{1em}
	Vice Diretor de Projetos: Ronivaldo D. Andrade \\
	\vspace{1em}
	\today
	}}}
	\begin{tikzpicture}[remember picture, overlay]
        \node[anchor=center] at ([xshift=5cm]current page.center) {
          \textcolor{lightpurple}{\fontsize{250}{300}\selectfont\textbf{]}}
        };
        \node[anchor=center] at ([xshift=3cm,yshift=1.5cm]current page.center) {
          \includegraphics[scale=0.16]{./assets/animate/cat/cat_000.png}
        };
      \end{tikzpicture}
\end{frame}
\end{document}