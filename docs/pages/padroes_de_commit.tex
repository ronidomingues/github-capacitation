\subsection{Padrões de Commits}

Manter um padrão consistente nas mensagens de commit é fundamental para garantir um histórico de versões claro, fácil de compreender e rastrear.  
Um bom padrão de commit permite que outros desenvolvedores entendam rapidamente \textbf{o que foi alterado}, \textbf{por que foi alterado} e \textbf{qual impacto a mudança traz}.

Existem diferentes convenções adotadas por equipes e comunidades. Veja a tabela \ref{tab:commit_patterns} na página \pageref{tab:commit_patterns} para ver padrões populares.

Acesse também as referencias:
\begin{itemize}
  \item \citeonline{adorno_commits};
  \item \citeonline{iuricode_commits};
\end{itemize}

\subsubsection*{Boas práticas ao escrever commits}
\begin{itemize}
  \item Use o modo \textbf{imperativo} (ex.: ``adiciona``, ``corrige``, ``remove``);
  \item Mantenha a linha de assunto com no máximo \textbf{50 caracteres};
  \item Separe título e corpo com uma \textbf{linha em branco};
  \item Descreva \textbf{o motivo da mudança} no corpo, não apenas o que foi alterado;
  \item Use \textbf{referências a issues} quando aplicável (ex.: ``Closes \#123``);
  \item Escreva mensagens em português ou inglês, mas mantenha um idioma único no projeto.
\end{itemize}

\subsubsection*{Exemplo de commit completo}
\begin{verbatim}
feat(api): adiciona endpoint para criação de pedidos

Adiciona o endpoint POST /orders para permitir o cadastro
de novos pedidos. Inclui validação de campos obrigatórios
e testes unitários. Closes #42.
\end{verbatim}

\subsubsection*{Vantagens de seguir um padrão}
\begin{itemize}
  \item Histórico limpo e fácil de entender;
  \item Facilita revisão de código e auditorias;
  \item Permite geração automática de changelogs;
  \item Ajuda em pipelines de CI/CD e versionamento semântico;
  \item Melhora colaboração em equipes e projetos open source.
\end{itemize}