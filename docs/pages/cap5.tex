\chapter{Git LFS}
\section{O que é?}

O \textbf{Git LFS (Large File Storage)} é uma extensão oficial do Git projetada para o gerenciamento de \textbf{arquivos grandes ou binários} que não são tratados de forma eficiente pelo Git tradicional.  
Em vez de armazenar o conteúdo completo desses arquivos no histórico do repositório, o Git LFS substitui o arquivo original por um \textbf{ponteiro leve} — um pequeno arquivo de texto contendo informações sobre o objeto real, como seu identificador (\textit{hash}) e tamanho.  
O conteúdo real é armazenado separadamente, em um \textbf{servidor LFS}, podendo estar hospedado no próprio provedor Git (como o GitHub, GitLab ou Bitbucket) ou em um servidor dedicado.

Essa estratégia mantém o repositório \textbf{mais leve e ágil}, reduzindo o tempo de \textit{clone}, \textit{checkout} e \textit{fetch}, além de facilitar o versionamento de arquivos que mudam com frequência, como imagens, vídeos, áudios, modelos de aprendizado de máquina, arquivos de design (\texttt{.psd}), pacotes compactados (\texttt{.zip}), entre outros.

\subsection{Motivos e Problemas que Resolve}

Por padrão, o Git não é otimizado para lidar com arquivos grandes ou binários, pois ele foi projetado para versionar \textbf{texto}, como código-fonte.  
Existem várias limitações e problemas ao tentar versionar arquivos grandes diretamente:

\begin{itemize}
  \item \textbf{Limite de tamanho:} serviços como o GitHub impõem limites de \textbf{100 MB por arquivo}, impedindo o envio de arquivos muito grandes via Git comum.
  \item \textbf{Histórico inflado:} cada nova versão de um arquivo binário é armazenada integralmente, sem compressão eficiente, o que faz o repositório crescer rapidamente.
  \item \textbf{Operações lentas:} com muitos arquivos grandes no histórico, operações como \texttt{git clone}, \texttt{git fetch} e \texttt{git checkout} tornam-se mais lentas.
  \item \textbf{Dificuldade de merge:} arquivos binários não podem ser mesclados (\textit{merge}) facilmente, aumentando o risco de conflitos.
\end{itemize}

O \textbf{Git LFS} resolve esses problemas ao:
\begin{itemize}
  \item Armazenar apenas \textbf{ponteiros leves} no repositório Git;
  \item Manter o conteúdo real em um \textbf{armazenamento separado}, acessível sob demanda;
  \item Permitir a \textbf{revogação ou substituição} de arquivos sem reescrever o histórico;
  \item Proporcionar uma experiência de versionamento \textbf{transparente}, pois os comandos Git (\texttt{add}, \texttt{commit}, \texttt{push}) continuam funcionando normalmente.
\end{itemize}

\subsection{Como Funciona}

O Git LFS utiliza um sistema de filtros configurados no Git:
\begin{itemize}
  \item O filtro \textbf{clean} atua ao adicionar um arquivo rastreado pelo LFS, substituindo seu conteúdo real por um ponteiro antes de armazená-lo no repositório.
  \item O filtro \textbf{smudge} atua durante o \texttt{checkout} ou \texttt{clone}, baixando automaticamente o arquivo real do servidor LFS e substituindo o ponteiro pelo conteúdo original no diretório de trabalho.
\end{itemize}

O arquivo versionado no Git contém apenas algo como:

\begin{verbatim}
version https://git-lfs.github.com/spec/v1
oid sha256:3b6f1a8a...
size 1258291
\end{verbatim}

Essas informações são suficientes para que o Git LFS localize e baixe o conteúdo correto quando necessário.

\subsection{Vantagens}

\begin{itemize}
  \item Mantém o repositório leve e rápido;
  \item Suporta arquivos grandes (acima de 100 MB);
  \item Permite versionamento de arquivos binários;
  \item Integra-se com GitHub, GitLab e Bitbucket;
  \item Possibilita bloqueio de arquivos (\textit{lock}) para evitar conflitos.
\end{itemize}

\subsection{Limitações}

\begin{itemize}
  \item O armazenamento LFS pode ter \textbf{cotas e custos adicionais} em serviços remotos;
  \item Necessita de \textbf{instalação e configuração} local (\texttt{git lfs install});
  \item Requer que o \textbf{servidor remoto suporte LFS};
  \item Para repositórios antigos, pode ser necessário \textbf{migrar o histórico}.
\end{itemize}

\subsection{Exemplo de Uso}

\begin{verbatim}
# Instala o Git LFS no sistema
git lfs install

# Define tipos de arquivo que serão rastreados pelo LFS
git lfs track "*.zip"
git lfs track "*.psd"

# Adiciona e versiona normalmente
git add .gitattributes
git add arquivo.zip
git commit -m "Adiciona arquivo grande com Git LFS"
git push origin main
\end{verbatim}

\subsection{Boas Práticas}

\begin{itemize}
  \item Configure o Git LFS \textbf{antes de adicionar arquivos grandes};
  \item Use o comando \texttt{git lfs track} para definir padrões de arquivos;
  \item Verifique o status com \texttt{git lfs status};
  \item Evite rastrear arquivos pequenos em grande quantidade;
  \item Monitore o uso de armazenamento e transferências.
\end{itemize}