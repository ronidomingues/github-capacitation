\chapter{Commits, Merges e Pull Requests}

\section{Introdução}
Commits, merges e pull requests são elementos centrais no fluxo de trabalho com Git e plataformas como GitHub.  
Um \textbf{commit} é uma unidade lógica de alteração no repositório; um \textbf{merge} integra mudanças de uma branch em outra; e um \textbf{pull request} (PR) é uma solicitação formal de revisão e integração de uma branch — normalmente usada em colaboração para revisar código, executar checks automáticos (CI) e registrar a decisão de integrar.

\section{Commits}

\subsection{O que é um commit}
Um commit registra o estado do diretório de trabalho (os arquivos staged) em um nó do histórico do Git. Cada commit tem um identificador (SHA), metadata (autor, data) e uma mensagem que descreve a mudança.

\subsection{Boas práticas de commits}
\begin{itemize}
  \item Faça commits atômicos: cada commit deve representar uma \emph{única} mudança lógica.  
  \item Mensagens claras: use assunto imperativo curto (\(\sim\)50 caracteres) + linha em branco + corpo explicativo se necessário (72 caracteres por linha).  
  \item Inclua referência a issues quando relevante (ex.: ``Closes \#42'').  
  \item Evite commitar arquivos gerados (binários, dependências) — use \texttt{.gitignore}.  
  \item Assine commits quando necessário: \texttt{git commit -S -m "..."} (GPG).
\end{itemize}

\subsection{Fazendo commits — passo a passo}
\begin{enumerate}
  \item Verificar o estado dos arquivos:
  \begin{verbatim}
  git status
  \end{verbatim}
  \item Ver as mudanças não staged:
  \begin{verbatim}
  git diff       # diferenças não adicionadas ao stage
  git diff --staged  # diferenças preparadas para commit
  \end{verbatim}
  \item Adicionar alterações ao stage (todo arquivo ou interativo):
  \begin{verbatim}
  git add caminho/arquivo.txt  # adiciona arquivo específico
  git add .                    # adiciona tudo (cuidado)
  git add -p                   # adiciona parcialmente (patch)
  \end{verbatim}
  \item Criar o commit com boa mensagem:
  \begin{verbatim}
  git commit -m "Assunto curto em imperativo"
  # ou para mensagem longa (editor):
  git commit
  \end{verbatim}
  Exemplo de mensagem:
  \begin{verbatim}
  Corrige cálculo de juros

  Ajusta a fórmula de cálculo para considerar juros compostos
  quando o período é maior que 12 meses. Testes unitários
  adicionados para cobrir casos de fronteira.
  \end{verbatim}
  \item Ver histórico resumido:
  \begin{verbatim}
  git log --oneline --graph --decorate --all
  \end{verbatim}
\end{enumerate}

\subsection{Editar o último commit / corrigir mensagens}
\begin{verbatim}
# alterar o conteúdo do último commit (já staged)
git commit --amend

# alterar apenas a mensagem do último commit
git commit --amend -m "Nova mensagem"
\end{verbatim}
\textbf{Atenção:} se o commit já foi enviado ao remoto, evite \texttt{--amend} sem combinar com a equipe — ele reescreve o histórico e exigirá push forçado.

\subsection{Desfazer / alterar staging}
\begin{verbatim}
git restore --staged arquivo.txt   # remove do stage, mantém
                                   # alteração no working tree

git restore arquivo.txt            # descarta alteração no
                                   # working tree (se não commitada)

git reset --soft HEAD~1            # desfaz último commit, mantendo
                                   # mudanças staged

git reset --mixed HEAD~1           # desfaz último commit, mantendo
                                   # mudanças no working tree
                                   # (unstaged)

git reset --hard HEAD~1            # desfaz último commit e 
                                   # descarta mudanças (CUIDADO)
\end{verbatim}

\subsection{Padrões de Commits}

Manter um padrão consistente nas mensagens de commit é fundamental para garantir um histórico de versões claro, fácil de compreender e rastrear.  
Um bom padrão de commit permite que outros desenvolvedores entendam rapidamente \textbf{o que foi alterado}, \textbf{por que foi alterado} e \textbf{qual impacto a mudança traz}.

Existem diferentes convenções adotadas por equipes e comunidades. Veja a tabela \ref{tab:commit_patterns} na página \pageref{tab:commit_patterns} para ver padrões populares.

Acesse também as referencias:
\begin{itemize}
  \item \citeonline{adorno_commits};
  \item \citeonline{iuricode_commits};
\end{itemize}

\subsubsection*{Boas práticas ao escrever commits}
\begin{itemize}
  \item Use o modo \textbf{imperativo} (ex.: ``adiciona``, ``corrige``, ``remove``);
  \item Mantenha a linha de assunto com no máximo \textbf{50 caracteres};
  \item Separe título e corpo com uma \textbf{linha em branco};
  \item Descreva \textbf{o motivo da mudança} no corpo, não apenas o que foi alterado;
  \item Use \textbf{referências a issues} quando aplicável (ex.: ``Closes \#123``);
  \item Escreva mensagens em português ou inglês, mas mantenha um idioma único no projeto.
\end{itemize}

\subsubsection*{Exemplo de commit completo}
\begin{verbatim}
feat(api): adiciona endpoint para criação de pedidos

Adiciona o endpoint POST /orders para permitir o cadastro
de novos pedidos. Inclui validação de campos obrigatórios
e testes unitários. Closes #42.
\end{verbatim}

\subsubsection*{Vantagens de seguir um padrão}
\begin{itemize}
  \item Histórico limpo e fácil de entender;
  \item Facilita revisão de código e auditorias;
  \item Permite geração automática de changelogs;
  \item Ajuda em pipelines de CI/CD e versionamento semântico;
  \item Melhora colaboração em equipes e projetos open source.
\end{itemize}

\section{Merges}

\subsection{Tipos de merge}
\begin{description}
  \item[Fast-forward] Se a branch destino não avançou desde que a feature foi criada, o Git apenas avança o ponteiro (sem novo commit de merge).
  \item[Merge commit] Cria um commit de merge que documenta a união de dois históricos (útil para preservar contexto de branch).
  \item[Rebase] Reaplica commits de uma branch sobre outra, produzindo um histórico linear (reduz ``ruído'' dos merges, mas reescreve histórico).
\end{description}

\subsection{Merge local com merge commit (passo a passo)}
\begin{enumerate}
  \item Atualize a branch principal:
  \begin{verbatim}
  git checkout main
  git pull origin main
  \end{verbatim}
  \item Mudar para a branch de feature (se necessário):
  \begin{verbatim}
  git checkout feature/minha-feature
  git pull origin feature/minha-feature
  \end{verbatim}
  \item Voltar para a branch de destino e fazer merge:
  \begin{verbatim}
  git checkout main
  git merge --no-ff feature/minha-feature
  # --no-ff força um commit de merge,
  # preservando histórico da branch
  \end{verbatim}
  \item Caso não haja conflitos, o Git criará o commit de merge automaticamente. Em caso de conflitos, siga a seção "Resolver conflitos" abaixo.
  \item Envie as mudanças para o remoto:
  \begin{verbatim}
  git push origin main
  \end{verbatim}
\end{enumerate}

\subsection{Rebase (passo a passo) — para um histórico linear}
\begin{enumerate}
  \item Atualize a base:
  \begin{verbatim}
  git checkout main
  git pull origin main
  \end{verbatim}
  \item Rebase da feature sobre a main:
  \begin{verbatim}
  git checkout feature/minha-feature
  git rebase main
  \end{verbatim}
  \item Resolva conflitos (se aparecerem), use \texttt{git rebase --continue} e, ao final:
  \begin{verbatim}
  # force push (com cuidado) após reescrever histórico
  git push --force-with-lease origin feature/minha-feature
  \end{verbatim}
  \item \textbf{Observação:} reescrever histórico (rebase + push forçado) requer coordenação com outros colaboradores.
\end{enumerate}

\subsection{Resolver conflitos — passo a passo}
\begin{enumerate}
  \item Ao encontrar conflito durante merge/rebase, o Git mostra arquivos conflitantes:
  \begin{verbatim}
  CONFLICT (content): Merge conflict in caminho/arquivo.txt
  \end{verbatim}
  \item Abra o arquivo e localize os marcadores:
  \begin{verbatim}
  <<<<<<< HEAD
  conteúdo na branch atual (main)
  =======
  conteúdo vindo da branch feature/minha-feature
  >>>>>>> feature/minha-feature
  \end{verbatim}
  \item Edite o arquivo para a versão desejada (manter, combinar ou reescrever o trecho).
  \item Marque o conflito como resolvido:
  \begin{verbatim}
  git add caminho/arquivo.txt
  # se for rebase:
  git rebase --continue
  # se for merge:
  git commit   # caso o Git não tenha criado o commit automaticamente
  \end{verbatim}
  \item Se quiser abortar a operação:
  \begin{verbatim}
  git merge --abort    # durante um merge
  git rebase --abort   # durante um rebase
  \end{verbatim}
\end{enumerate}

\subsection{Squash e reescrita de commits (passo a passo)}
\begin{enumerate}
  \item Interative rebase para combinar commits locais:
  \begin{verbatim}
  git checkout feature/minha-feature
  git rebase -i main
  \end{verbatim}
  \item No editor que abre, marque \texttt{squash} (ou \texttt{s}) nos commits que deseja unir ao commit anterior. Salve e feche.
  \item Após o rebase, force push com segurança:
  \begin{verbatim}
  git push --force-with-lease origin feature/minha-feature
  \end{verbatim}
  \item Use \texttt{--force-with-lease} em vez de \texttt{--force} quando possível — ele evita sobrescrever pushes alheios.
\end{enumerate}

\section{Pull Requests (PR)}

\subsection{O que é um Pull Request}
Um PR é uma solicitação para integrar as mudanças de uma branch em outra (normalmente de uma branch de feature para \texttt{main} ou \texttt{develop}) e serve como ponto central para revisão de código, execução de pipelines de CI e documentação da mudança.

\subsection{Fluxo básico — criando um PR (via web)}
\begin{enumerate}
  \item Crie uma branch local para a sua feature:
  \begin{verbatim}
  git checkout -b feature/minha-feature
  \end{verbatim}
  \item Faça commits locais e envie a branch para o remoto:
  \begin{verbatim}
  git push -u origin feature/minha-feature
  \end{verbatim}
  \item No repositório do GitHub, acesse \texttt{Pull requests} \(\rightarrow\) \texttt{New pull request}.
  \item Selecione a \textbf{branch de origem} (feature/minha-feature) e a \textbf{branch de destino} (ex.: main).
  \item Preencha o título e a descrição: explique o \textbf{porquê} e o \textbf{o que} foi alterado. Use referências a issues (ex.: ``Closes \#123'').
  \item Configure revisores, labels, milestone e assignees.
  \item Se ainda não está pronta, crie como \textbf{Draft pull request} (rascunho).
  \item Aguarde revisão, corrija comentários fazendo novos commits na mesma branch e push — eles serão anexados automaticamente ao PR.
  \item Quando aprovado, escolha a estratégia de merge (merge commit / squash and merge / rebase and merge) e realize o merge.
\end{enumerate}

\subsection{Criar e gerenciar PRs via GitHub CLI (passo a passo)}
\begin{verbatim}
# autentique (uma vez)
gh auth login

# depois de push da branch
gh pr create --base main --head feature/minha-feature --title "Título do PR" --body "Descrição detalhada" --reviewer usuario1,usuario2

# abrir PR como draft:
gh pr create --draft --base main --head feature/minha-feature --fill

# listar PRs locais:
gh pr list

# fechar ou mesclar via CLI:
gh pr merge <numero-ou-url> --squash --delete-branch
# ou
gh pr merge <numero> --merge      # cria commit de merge
gh pr merge <numero> --rebase     # rebase and merge
\end{verbatim}

\subsection{Checklist para revisão de Pull Request}
\begin{itemize}
  \item O PR tem um título e descrição claros (o \emph{porquê} e o \emph{o que});  
  \item Testes automatizados adicionados/atualizados e pipeline CI passando;  
  \item Código segue padrões de lint e estilo;  
  \item Mudanças pequenas e focadas (um PR por responsabilidade);  
  \item Documentação e comentários quando necessário;  
  \item Evidências visuais (screenshots) quando há alterações de UI.  
\end{itemize}

\subsection{Depois do merge — limpeza e sincronização}
\begin{verbatim}
# atualizar a branch principal local
git checkout main
git pull origin main

# remover branch remota (após merge)
git push origin --delete feature/minha-feature

# remover branch local
git branch -d feature/minha-feature

# caso a branch não possa ser deletada por não estar totalmente mesclada:
git branch -D feature/minha-feature
\end{verbatim}
Também é útil rodar:
\begin{verbatim}
git fetch --prune
\end{verbatim}
para remover referências remotas deletadas.

\section{Boas práticas e recomendações finais}
\begin{itemize}
  \item Mantenha PRs pequenos e revisáveis; grandes PRs demoram mais para receber feedback.  
  \item Automatize checks com CI (testes, lint, análise estática) e impeça merge enquanto falharem (branch protection).  
  \item Use \texttt{--force-with-lease} quando precisar reescrever histórico; nunca force sem checar se colegas não empurraram commits.  
  \item Documente o fluxo do time (merge strategy preferida — ex.: squash para commits limpos, merge commit para histórico preservado).  
  \item Escreva mensagens de commit úteis e mantenha um padrão no time (ex.: Conventional Commits) para facilitar geração automática de changelogs.
\end{itemize}

\section{Exemplos rápidos de comandos úteis}
\begin{verbatim}
# ver status e diferenças
git status
git diff
git diff --staged

# histórico e log
git log --oneline --graph --decorate --all

# adicionar parcialmente
git add -p

# rebase interativo (últimos 3 commits)
git rebase -i HEAD~3

# desfazer mudanças locais (trabalhe com cuidado)
git restore arquivo.txt
git reset --hard HEAD

# forçar push com segurança
git push --force-with-lease origin minha-branch
\end{verbatim}

% Fim do capítulo