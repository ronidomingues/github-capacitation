\chapter{Conclusão}

Ao longo desta capacitação, foram abordados os principais conceitos e ferramentas que compõem o ecossistema do GitHub, desde a criação e configuração de repositórios até a realização de operações complexas como merges, rebases, pull requests e a automação de pipelines com GitHub Actions.

O domínio dessas habilidades não apenas facilita a colaboração em projetos de software, mas também promove a adoção de boas práticas de desenvolvimento, como commits semânticos, revisão de código e integração contínua. A utilização de recursos como Git LFS para arquivos grandes e a assinatura de commits com chaves GPG reforça a segurança e a integridade do versionamento.

Por fim, a realização dos exercícios práticos propostos consolida o aprendizado e prepara o participante para atuar em ambientes reais, contribuindo de forma eficiente e profissional em projetos individuais e em equipe. Espera-se que este material sirva como referência contínua e incentive a adoção de um fluxo de trabalho organizado, colaborativo e alinhado com as melhores práticas do mercado.