% ----------------------------------------------------------------------------------------------------------------------
% Definindo minhas cores bases;
% ----------------------------------------------------------------------------------------------------------------------
\definecolor{branco}{HTML}{FFFFFF}
\definecolor{roxoescuro}{HTML}{1E1647}
\definecolor{roxoclaro}{HTML}{614AD3}
\definecolor{amarelo}{HTML}{FFFF00}
% ----------------------------------------------------------------------------------------------------------------------
% Configurações para ambiente de código;
% ----------------------------------------------------------------------------------------------------------------------
\lstdefinestyle{custom}{
    backgroundcolor=\color{branco},
    basicstyle=\footnotesize\ttfamily,
    breaklines=true,
    frame=single,
    rulecolor=\color{roxoescuro},
    upquote=true
}
% ----------------------------------------------------------------------------------------------------------------------
% Configurações de Links;
% ----------------------------------------------------------------------------------------------------------------------
\hypersetup{
    colorlinks=true,
    linkcolor=roxoescuro,
    filecolor=amarelo,
    urlcolor=roxoclaro,
    pdftitle={Guia Prático para a Capacitação em GitHub},
    pdfpagemode=FullScreen,
    pdfauthor={Ronivaldo Domingues de Andrade},
    pdfsubject={Capacitação em GitHub},
    pdfkeywords={GitHub, Controle de Versão, Colaboração, Repositórios, Branches, Commits, Pull Requests, CI/CD}
}
% ----------------------------------------------------------------------------------------------------------------------
% Configurações do lstset do listings
% ----------------------------------------------------------------------------------------------------------------------
\lstset{
    backgroundcolor=\color{gray!10},
    basicstyle=\ttfamily\footnotesize,
    breaklines=true,
    frame=single,
    numbers=left,
    numberstyle=\tiny\color{gray},
    captionpos=b
}