\chapter{Padrões de Commits}
\label{ap:padroes_de_commits}

Este apêndice apresenta uma referência completa dos principais padrões de commits organizados por categoria e funcionalidade.

\section*{Padrões de Commits}

\begin{longtable}{|p{0.12\textwidth}|p{0.2\textwidth}|p{0.25\textwidth}|p{0.3\textwidth}|}
  \caption{Padrões de Commits Profissionais} \label{tab:commit_patterns} \\
  \hline
  \textbf{Tipo} & \textbf{Descrição} & \textbf{Quando Utilizar} & \textbf{Exemplo} \\
  \hline
  \endfirsthead
  \multicolumn{4}{c}%
  {\textbf{Continuação da Tabela: Padrões de Commits Profissionais}} \\
  \hline
  \textbf{Tipo} & \textbf{Descrição} & \textbf{Quando Utilizar} & \textbf{Exemplo} \\
  \hline
  \endhead
  \multicolumn{4}{|r|}{Continua na próxima página} \\
  \hline
  \endfoot
  \endlastfoot

  feat & 
  Introduz uma nova funcionalidade ao projeto. &
  Quando adicionar novas capacidades ou funcionalidades. &
  \texttt{feat: adicionar autenticação via OAuth2} \\
  & & & \texttt{feat(api): implementar endpoint de usuários} \\
  \hline

  fix & 
  Corrige um bug ou erro no código. &
  Quando resolver problemas ou defeitos no sistema. &
  \texttt{fix: corrigir cálculo de impostos} \\
  & & & \texttt{fix(auth): resolver loop infinito no login} \\
  \hline

  docs & 
  Alterações na documentação. &
  Quando atualizar README, comentários ou documentação. &
  \texttt{docs: atualizar guia de instalação} \\
  & & & \texttt{docs(api): adicionar exemplos de uso} \\
  \hline

  style & 
  Mudanças que não afetam o significado do código. &
  Ao ajustar formatação, espaços, vírgulas, etc. &
  \texttt{style: corrigir indentação no CSS} \\
  & & & \texttt{style: remover espaços em branco} \\
  \hline

  refactor & 
  Reestruturação do código sem alterar comportamento. &
  Quando melhorar a estrutura sem mudar funcionalidades. &
  \texttt{refactor: extrair método para reduzir complexidade} \\
  & & & \texttt{refactor(db): otimizar queries SQL} \\
  \hline

  perf & 
  Melhorias de performance. &
  Ao otimizar velocidade ou eficiência do código. &
  \texttt{perf: otimizar algoritmo de ordenação} \\
  & & & \texttt{perf: reduzir tempo de carregamento em 30\%} \\
  \hline

  test & 
  Adiciona ou modifica testes. &
  Ao criar novos testes ou corrigir existentes. &
  \texttt{test: adicionar testes unitários para UserService} \\
  & & & \texttt{test: corrigir teste de integração} \\
  \hline

  build & 
  Mudanças no sistema de build ou dependências. &
  Ao atualizar dependências, Webpack, Maven, etc. &
  \texttt{build: atualizar React para v18} \\
  & & & \texttt{build: configurar Dockerfile} \\
  \hline

  ci & 
  Mudanças na configuração de CI/CD. &
  Ao modificar GitHub Actions, GitLab CI, Jenkins, etc. &
  \texttt{ci: adicionar pipeline de deploy automático} \\
  & & & \texttt{ci: configurar testes E2E no GitHub Actions} \\
  \hline

  chore & 
  Tarefas de manutenção e rotina. &
  Para atualizações de rotina que não se encaixam em outras categorias. &
  \texttt{chore: atualizar versão do package.json} \\
  & & & \texttt{chore: limpar dependências não utilizadas} \\
  \hline

  revert & 
  Reverte um commit anterior. &
  Quando necessário desfazer mudanças anteriores. &
  \texttt{revert: "feat: adicionar feature X"} \\
  & & & \texttt{revert: commit abc1234} \\
  \hline

  hotfix & 
  Correção crítica para produção. &
  Para bugs críticos que exigem correção imediata. &
  \texttt{hotfix: corrigir vulnerabilidade de segurança} \\
  & & & \texttt{hotfix: resolver falha no processamento de pagamentos} \\
  \hline

  security & 
  Correções relacionadas à segurança. &
  Ao abordar vulnerabilidades ou melhorar segurança. &
  \texttt{security: atualizar bibliotecas com vulnerabilidades} \\
  & & & \texttt{security: implementar sanitização de inputs} \\
  \hline

  init & 
  Commit inicial do projeto. &
  Para o primeiro commit de um novo projeto. &
  \texttt{init: configuração inicial do projeto} \\
  & & & \texttt{init: estrutura base da aplicação} \\
  \hline
\end{longtable}