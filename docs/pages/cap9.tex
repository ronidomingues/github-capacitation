\chapter{Exercícios Práticos}

\textbf{Observação:} Todas as atividades devem seguir as boas práticas de commits, merges e pull requests.

\section{Git e GitHub-CLI}

\subsection{Objetivo}
Verificar se as ferramentas estão instaladas corretamente.

\subsection{Passo a Passo Detalhado}

\begin{enumerate}
    \item \textbf{Abrir o terminal} (Prompt de Comando no Windows, Terminal no Mac/Linux)
    
    \item \textbf{Verificar se o Git está instalado:}
    \begin{verbatim}
    git --version
    \end{verbatim}
    \textit{Resultado esperado:} Deve aparecer algo como \texttt{git version 2.xx.x}
    
    \item \textbf{Verificar se o GitHub CLI está instalado:}
    \begin{verbatim}
    gh --version
    \end{verbatim}
    \textit{Resultado esperado:} Deve aparecer algo como \texttt{gh version 2.xx.x}
\end{enumerate}

\subsection{Problemas Comuns e Soluções}
\begin{itemize}
    \item Se algum comando não for reconhecido, reinstale a ferramenta
    \item No Windows, talvez seja necessário reiniciar o computador após a instalação
\end{itemize}

\section{Criar um repositório no GitHub via CLI}

\subsection{Objetivo}
Criar um repositório público com descrição.

\subsection{Pré-requisito}
Fazer login no GitHub CLI:
\begin{verbatim}
gh auth login
\end{verbatim}

\subsection{Passo a Passo}
\begin{enumerate}
    \item \textbf{Criar o repositório:}
    \begin{verbatim}
    gh repo create meu-primeiro-repo --public 
    --description "Meu primeiro repositório" --clone
    \end{verbatim}
    
    \item \textbf{Entrar na pasta do repositório:}
    \begin{verbatim}
    cd meu-primeiro-repo
    \end{verbatim}
\end{enumerate}

\subsection{README.md}

\subsubsection{O que é README.md}
\begin{itemize}
    \item É a "cara" do seu projeto no GitHub
    \item Explica o que seu projeto faz, como usar, etc.
    \item Usa uma linguagem chamada Markdown (por isso o .md)
\end{itemize}

\subsubsection{Como Criar}
\begin{enumerate}
    \item \textbf{Criar o arquivo:}
    \begin{verbatim}
    echo "# Meu Primeiro Projeto" >> README.md
    \end{verbatim}
    
    \item \textbf{Adicionar conteúdo:}
    \begin{verbatim}
    # Meu Primeiro Projeto
     
    Este é meu primeiro repositório no GitHub!
     
    ## O que este projeto faz?
     
    - Aprender Git e GitHub
    - Praticar comandos
    - Compartilhar conhecimento
     
    ## Como usar?
     
    1. Clone este repositório
    2. Siga as instruções
    3. Aprenda!
    \end{verbatim}
    
    \item \textbf{Salvar e enviar para o GitHub:}
    \begin{verbatim}
    git add README.md
    git commit -m "docs(readme): Adiciona README com 
    a descrição do projeto"
    git push origin main
    \end{verbatim}
    Em \texttt{git push origin main}, pode ser colocado a flag \texttt{-u}, ou seja, \texttt{git push -u origin main} isso precisará ser feito apenas uma vez os próximos pushes poderão apenas se fazer com \texttt{git push}.
    A flag -u faz com que o git se torne um colaborador do repositório, ou seja, ele não precisa mais digitar origin e main, ele já sabe que é o repositório principal.
\end{enumerate}

\subsection{LICENSE}

\subsubsection{Por que usar LICENSE}
\begin{itemize}
    \item Define como outras pessoas podem usar seu código
    \item Protege seus direitos autorais
    \item Torna seu projeto mais profissional
\end{itemize}

\subsubsection{Como Adicionar Licença MIT}
\begin{enumerate}
    \item \textbf{Criar arquivo LICENSE:}
    \begin{verbatim}
    touch LICENSE
    \end{verbatim}
    
    \item \textbf{Adicionar conteúdo da licença MIT:}
    \begin{lstlisting}[caption={Licença MIT},label=lst:mit-license]
    MIT License

    Copyright (c) [ano] [seu nome]

    Permission is hereby granted, free of charge, to any person 
    obtaining a copy of this software and associated documentation 
    files (the "Software"), to deal in the Software without 
    restriction, including without limitation the rights to use, 
    copy, modify, merge, publish, distribute, sublicense, and/or 
    sell copies of the Software, and to permit persons to whom 
    the Software is furnished to do so, subject to the following 
    conditions:

    The above copyright notice and this permission notice shall 
    be included in all copies or substantial portions of the Software.

    THE SOFTWARE IS PROVIDED "AS IS", WITHOUT WARRANTY OF ANY KIND, 
    EXPRESS OR IMPLIED, INCLUDING BUT NOT LIMITED TO THE WARRANTIES 
    OF MERCHANTABILITY, FITNESS FOR A PARTICULAR PURPOSE AND 
    NONINFRINGEMENT. IN NO EVENT SHALL THE AUTHORS OR COPYRIGHT 
    HOLDERS BE LIABLE FOR ANY CLAIM, DAMAGES OR OTHER LIABILITY, 
    WHETHER IN AN ACTION OF CONTRACT, TORT OR OTHERWISE, ARISING 
    FROM, OUT OF OR IN CONNECTION WITH THE SOFTWARE OR THE USE OR 
    OTHER DEALINGS IN THE SOFTWARE.
    \end{lstlisting}
    
    \item \textbf{Substituir \texttt{[ano]} e \texttt{[seu nome]} pelos seus dados}
    
    \item \textbf{Salvar e enviar:}
    \begin{verbatim}
    git add LICENSE
    git commit -m "chore(licensing):Adiciona licença MIT"
    git push origin main
    \end{verbatim}
\end{enumerate}

\section{Clonando um repositório do GitHub}

\subsection{Objetivo}
Aprender a baixar repositórios existentes.

\subsection{Passo a Passo}
\begin{enumerate}
    \item \textbf{Repositório alvo: https://github.com/ronidomingues/github-capacitation}
    
    \item \textbf{Copiar a URL do repositório}
    
    \item \textbf{No terminal, clonar:}
    \begin{verbatim}
    git clone https://github.com/ronidomingues/
    github-capacitation.git
    \end{verbatim}
    
    \item \textbf{Entrar na pasta criada:}
    \begin{verbatim}
    cd github-capacitation
    \end{verbatim}
    
    \item \textbf{Verificar o conteúdo:}
    \begin{verbatim}
    ls -la
    \end{verbatim}
\end{enumerate}

\section{Github Pages}

\subsection{Objetivo}
Publicar um site gratuitamente.

\subsection{Passo a Passo}
\begin{enumerate}
    \item \textbf{Criar novo repositório:}
    \begin{verbatim}
    gh repo create meu-site --public 
    --description "Meu primeiro site" --clone
    cd meu-site
    \end{verbatim}
    
    \item \textbf{Copiar os arquivos do jogo} (HTML, CSS, JS) para a pasta do repositório
    
    \item \textbf{Verificar estrutura:}
    \begin{verbatim}
    ls -la
    \end{verbatim}
    
    \item \textbf{Adicionar, commitar e enviar:}
    \begin{verbatim}
    git add .
    git commit -m "feat(jogo): adiciona arquivos do jogo"
    git push origin main
    \end{verbatim}
    
    \item \textbf{Ativar GitHub Pages:}
    \begin{itemize}
        \item No GitHub, vá em \textbf{Settings} → \textbf{Pages}
        \item Em \textbf{Source}, selecione \textbf{main branch}
        \item Clique \textbf{Save}
    \end{itemize}
    
    \item \textbf{Acessar seu site:}
    \begin{itemize}
        \item URL será: \texttt{https://seu-usuario.github.io/meu-site}
    \end{itemize}
\end{enumerate}

\section{Github Actions}

\subsection{Objetivo}
Automatizar execução de código Python.

\subsection{Passo a Passo}
\begin{enumerate}
    \item \textbf{Criar repositório para o código Python:}
    \begin{verbatim}
    gh repo create meu-script-python --public 
    --description "Script Python com GitHub Actions" --clone
    cd meu-script-python
    \end{verbatim}
    
    \item \textbf{Copiar o arquivo Python fornecido para o repositório}
    
    \item \textbf{Criar pasta para workflows:}
    \begin{verbatim}
    mkdir -p .github/workflows
    \end{verbatim}
    
    \item \textbf{Criar arquivo de workflow:}
    \begin{verbatim}
    touch .github/workflows/python.yml
    \end{verbatim}
    
    \item \textbf{Adicionar conteúdo ao workflow - Prencher o que falta:}
    \par
    Há um miodelo com \texttt{fortran 90}, disponível na pasta materials.
    \begin{verbatim}
        name: "Executar script Python e Commitar resultado"

        on:
          push:
            branches: [ main ]
          workflow_dispatch:

        jobs:
          build-run:
            runs-on: ubuntu-latest
            steps:
                # 1️⃣ Faz o clone do repositório para a VM Ubuntu;
                # 2️⃣ Configura o Python a ser usado pela VM;
                -name: Instalar Python3
                uses: actions/setup-python@v5
                with:
                  python-version: '3.x'
                # 3️⃣ Executa o script Python;
                # 4️⃣ Cria um commit com o resultado;
                -name: Commitar PDFs gerados
                run: |
                    git config user.name "github-actions[bot]"
                    git config user.email "github-actions
                    [bot]@users.noreply.github.com"
                    git add materials/*.pdf
                    git commit -m "Atualizar PDFs compilados
                    automaticamente [skip ci]" || echo "Nenhuma
                    alteração para commitar"
                    git push
                env:
                    GITHUB_TOKEN: ${{ secrets.GITHUB_TOKEN }}
    \end{verbatim}
    \subsection{Uso de \texttt{[skip ci]} no GitHub Actions}

No GitHub Actions, um workflow normalmente é disparado por eventos como:

\begin{verbatim}
on:
  push:
    branches:
      - main
\end{verbatim}

Ou seja, \textbf{cada \texttt{git push}} aciona o workflow.  

Quando o próprio workflow realiza um commit e push automaticamente (por exemplo, atualizando arquivos gerados ou listas), isso poderia disparar o workflow novamente, criando um \textbf{loop infinito}.  

Para evitar esse problema, é possível incluir no commit uma anotação especial:

\begin{verbatim}
git commit -m "Atualiza lista automática [skip ci]"
\end{verbatim}

O código \texttt{[skip ci]} instrui o GitHub Actions (e outros sistemas de CI, como GitLab CI ou Travis CI) a \textbf{ignorar este commit}, ou seja, não disparar nenhum workflow.  

Dessa forma, o workflow pode atualizar arquivos ou fazer commits automaticamente sem reiniciar seu próprio processo indefinidamente.  

\textbf{Observação:} Além de \texttt{[skip ci]}, também é possível usar \texttt{[ci skip]}, que possui a mesma função.

    \item \textbf{Adicionar, commitar e enviar tudo:}
    \begin{verbatim}
    git add .
    git commit -m "feat:Adiciona script Python e GitHub Actions"
    git push origin main
    \end{verbatim}
    
    \item \textbf{Verificar execução:}
    \begin{itemize}
        \item No GitHub, vá em \textbf{Actions} para ver o workflow rodando
    \end{itemize}
\end{enumerate}

\section{Merge}

\subsection{Objetivo}
Aprender a juntar alterações de diferentes origens.

\subsection{Passo a Passo}
\begin{enumerate}
    \item \textbf{Fazer alteração REMOTA:}
    \begin{itemize}
        \item No GitHub, edite o README.md online
        \item Adicione uma linha no final
        \item Commit a alteração
    \end{itemize}
    
    \item \textbf{Fazer alteração LOCAL:}
    \begin{verbatim}
    # No seu computador, no mesmo repositório
    echo "Alteração local" >> arquivo-local.txt
    git add arquivo-local.txt
    git commit -m "Adiciona arquivo local"
    \end{verbatim}
    
    \item \textbf{Tentar enviar alteração local:}
    \begin{verbatim}
    git push origin main
    # VAI DAR ERRO! Porque tem alteração remota 
    # que você não tem localmente
    \end{verbatim}
    
    \item \textbf{Fazer merge:}
    \begin{verbatim}
    git pull origin main
    # Isso baixa as alterações remotas e faz merge 
    # com suas alterações locais
    \end{verbatim}
    
    \item \textbf{Resolver conflitos (se houver):}
    \begin{itemize}
        \item Se Git não conseguir juntar automaticamente, ele pedirá para resolver manualmente
        \item Abra os arquivos com conflitos, resolva e depois:
        \begin{verbatim}
        git add .
        git commit -m "Resolve conflitos de merge"
        git push origin main
        \end{verbatim}
    \end{itemize}
\end{enumerate}

\section{Pull Request}

\subsection{Objetivo}
Contribuir para projetos de outras pessoas.

\subsection{Passo a Passo}
\begin{enumerate}
    \item \textbf{Fork do repositório original:}
    \begin{itemize}
        \item No GitHub, vá para o repositório \url{https://github.com/ronidomingues/github-capacitation}
        \item Clique em \textbf{Fork} (canto superior direito)
        \item Isso cria uma cópia em sua conta
    \end{itemize}
    
    \item \textbf{Clonar SEU fork:}
    \begin{verbatim}
    git clone https://github.com/seu-usuario/
    repositorio-forkado.git
    cd repositorio-forkado
    \end{verbatim}
    
    \item \textbf{Criar branch para sua feature:}
    \begin{verbatim}
    git checkout -b minha-feature
    \end{verbatim}
    
    \item \textbf{Fazer suas alterações:}
    \par Entre na pasta \texttt{presences} e adicione um arquivo \texttt{.txt} com o seu nome, por exemplo \texttt{roni.txt}, esse arquivo não precisa ter nenhum conteúdo, mas se queiser deixar uma avaliação de tudo até aqui será ótimo $\left. ;- \right)$.
    \begin{verbatim}
    echo "Minha avaliacao" >> presences/meu-nome.txt
    \end{verbatim}
    
    \item \textbf{Commit e push:}
    \begin{verbatim}
    git add .
    git commit -m "<tipo>(escopo): <descrição>"
    git push origin minha-feature
    \end{verbatim}
    
    \item \textbf{Criar Pull Request:}
    \begin{itemize}
        \item No GitHub, vá para SEU fork
        \item Clique em \textbf{Pull Request} → \textbf{New Pull Request}
        \item Selecione: base (repositório original) ← compare (sua branch)
        \item Descreva suas alterações
        \item Clique \textbf{Create Pull Request}
    \end{itemize}
\end{enumerate}

\section{Materiais de Apoio}

\subsection{Checklist para Cada Exercício}
\begin{itemize}
    \item [$\square$] Comandos executados sem erro
    \item [$\square$] Arquivos criados corretamente
    \item [$\square$] Commits com mensagens descritivas
    \item [$\square$] Push realizado com sucesso
    \item [$\square$] Resultado verificado no GitHub
\end{itemize}

\subsection{Comandos Úteis para Consulta}
\begin{verbatim}
# Status do repositório
git status

# Ver histórico de commits
git log --oneline

# Ver diferenças
git diff

# Ver configuração
git config --list
\end{verbatim}

\subsection{Dicas para Boas Práticas}
\begin{itemize}
    \item Commits frequentes e pequenos
    \item Mensagens de commit claras e descritivas
    \item Sempre fazer pull antes de push
    \item Testar localmente antes de enviar
    \item Revisar código antes de criar PR
\end{itemize}