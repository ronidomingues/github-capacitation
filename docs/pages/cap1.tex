% ----------------------------------------------------------------------------------------------------------------------
% Início dos capítulos do trabalho;
% ----------------------------------------------------------------------------------------------------------------------
\chapter{Introdução}

O GitHub consolidou-se como uma das principais plataformas de desenvolvimento colaborativo, sendo amplamente adotado por equipes e desenvolvedores individuais para o controle de versão, a gestão de projetos e a integração contínua. No entanto, o uso eficiente de suas ferramentas exige não apenas familiaridade com conceitos básicos, mas também o domínio de boas práticas e fluxos de trabalho modernos.

Esta capacitação foi elaborada com o objetivo de oferecer um guia prático e acessível para o uso do GitHub e de suas tecnologias associadas, como Git, GitHub CLI, Git LFS, GitHub Pages e GitHub Actions. O material abrange desde a configuração inicial do ambiente até a execução de operações avançadas, como a assinatura de commits com GPG e a automação de fluxos de trabalho.

Além disso, são apresentados exercícios práticos que simulam situações reais de desenvolvimento, permitindo que os participantes vivenciem todo o ciclo de colaboração em projetos versionados. Com isso, espera-se que, ao final do curso, os participantes estejam aptos a contribuir de forma segura, organizada e profissional em repositórios locais e remotos, seja em projetos pessoais ou corporativos.