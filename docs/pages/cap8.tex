\chapter{Assinaturas de Commits com chave GPG}

\section{O que é?}
A assinatura de commits com \textbf{GPG (GNU Privacy Guard)} é um recurso do Git que permite garantir a autenticidade e a integridade das alterações feitas em um repositório. Quando um commit é assinado, outras pessoas podem verificar que aquele commit foi realmente feito por você, evitando alterações fraudulentas ou commits não autorizados.

\subsection{Importância das assinaturas GPG}
\begin{itemize}
    \item \textbf{Segurança:} Confirma que o commit foi feito pelo autor legítimo.
    \item \textbf{Integridade:} Permite verificar se o commit não foi alterado após sua criação.
    \item \textbf{Transparência:} Em projetos open source, facilita identificar contribuições confiáveis.
\end{itemize}

\section{Como usar?}

\subsection{Passo 1: Instalar o GPG}
Antes de assinar commits, você precisa instalar o GPG no seu sistema.

\begin{itemize}
    \item \textbf{Linux/Debian:}
    \begin{verbatim}
sudo apt update
sudo apt install gnupg
    \end{verbatim}

    \item \textbf{Windows:} Instale o Gpg4win (\url{https://www.gpg4win.org/})

    \item \textbf{MacOS:} 
    \begin{verbatim}
brew install gnupg
    \end{verbatim}
\end{itemize}

\subsection{Passo 2: Gerar uma chave GPG}
Para criar uma chave GPG pessoal, use o comando:

\begin{verbatim}
gpg --full-generate-key
\end{verbatim}

O terminal fará algumas perguntas:
\begin{enumerate}
    \item Tipo de chave: selecione \texttt{RSA and RSA (default)}.
    \item Tamanho da chave: recomendo \texttt{4096 bits} para maior segurança.
    \item Validade da chave: escolha o período de validade ou \texttt{0} para sem expiração.
    \item Nome e e-mail: use o mesmo e-mail configurado no Git (\texttt{git config user.email}).
    \item Senha: defina uma senha segura para proteger sua chave.
\end{enumerate}

\subsection{Passo 3: Listar chaves e copiar o ID da chave}
Para ver as chaves criadas:

\begin{verbatim}
gpg --list-secret-keys --keyid-format LONG
\end{verbatim}

O resultado terá um formato como:
\begin{verbatim}
sec   rsa4096/ABCDEF1234567890 2025-01-01 [SC]
      Key fingerprint = 1234 5678 9ABC DEF0 1234
      5678 9ABC DEF0 1234 5670
uid                 Seu Nome <seuemail@example.com>
\end{verbatim}

O que você precisa é do \textbf{ID da chave}, que no exemplo acima é \texttt{ABCDEF1234567890}.

\subsection{Passo 4: Configurar o Git para usar a chave GPG}
Diga ao Git qual chave usar para assinar commits:

\begin{verbatim}
git config --global user.signingkey ABCDEF1234567890
git config --global commit.gpgsign true
\end{verbatim}

\subsection{Passo 5: Adicionar a chave GPG ao GitHub}
Para que o GitHub reconheça seus commits assinados:
\begin{enumerate}
    \item Copie a chave pública:
    \begin{verbatim}
gpg --armor --export ABCDEF1234567890
    \end{verbatim}
    \item Entre no GitHub: \texttt{Settings > SSH and GPG keys > New GPG key}
    \item Cole a chave pública e salve.
\end{enumerate}

\subsection{Passo 6: Fazer commits assinados}
A partir de agora, todos os commits serão assinados automaticamente. Você também pode assinar commits individualmente:

\begin{verbatim}
git commit -S -m "Mensagem do commit"
\end{verbatim}

No GitHub, os commits assinados aparecerão com a etiqueta \textbf{Verified}.

\subsection{Passo 7: Verificar commits assinados}
Para verificar um commit localmente, use:

\begin{verbatim}
git log --show-signature
\end{verbatim}

O Git mostrará se o commit foi assinado corretamente e qual chave foi usada.

\section{Dicas de segurança e boas práticas}
\begin{itemize}
    \item Proteja sua chave GPG com uma senha forte.
    \item Faça backup da chave privada em um local seguro.
    \item Não compartilhe sua chave privada.
    \item Rotacione suas chaves periodicamente, se necessário.
\end{itemize}
